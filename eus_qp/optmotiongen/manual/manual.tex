%\documentstyle[art10,titlepage,makeidx,twoside,EPSF/epsf,mytabbing]{j-article}

% euslisp
\newif\ifeuslisp
\euslisptrue

%%% added 2004.12.14
\documentclass[]{jarticle}
\usepackage{makeidx,mytabbing,fancyheadings}


\usepackage[dvipdfmx]{graphicx,color,epsfig}
\let\epsfile=\epsfig
\usepackage[dvipdfmx,bookmarks=true,bookmarksnumbered=true,bookmarkstype=toc]{hyperref}
\ifnum 42146=\euc"A4A2 \AtBeginDvi{\special{pdf:tounicode EUC-UCS2}}\else
\AtBeginDvi{\special{pdf:tounicode 90ms-RKSJ-UCS2}}\fi

%%%
\newcommand{\eusversion}{9.00}
\newcommand{\irteusversion}{1.00}


\flushbottom
\makeindex
\pagestyle{myheadings}
\oddsidemargin=0cm
\evensidemargin=0cm

% A4 size
\textwidth=16.5cm
\textheight=24.6cm
\topmargin=-0.8cm
\oddsidemargin= 0.5cm
\evensidemargin=0.5cm

% Letter size
%\topmargin=0.5cm
%\textwidth=17.6cm
%\textheight=23cm
%\oddsidemargin= 0.2cm
%\evensidemargin=0.2cm

\parindent=10pt
\parskip=1mm
%\baselineskip 14pt

\setcounter{totalnumber}{3}
\renewcommand{\topfraction}{0.99}       % 85% of a page (from page
                                        % top)
                                        % can be occupied by tbl / fig
                                        %
\renewcommand{\bottomfraction}{0.99}    % 85% of a page (from page
                                        % bottom)
                                        % can be occupied by tbl / fig
\renewcommand{\textfraction}{0.0}      % text shoud occupy 15% or
                                        % more in
                                        % a page
%\renewcommand{\floatpagefraction}{0.99} % 70% or more shoud occupy in
                                        % a
                                        % a float page


%%% removed 2004.12.14 \jintercharskip=0pt plus3.0pt minus1pt%

\usepackage{amsmath,amssymb}
\usepackage{arydshln}
\usepackage{mathrsfs}
\usepackage{cases} %% subnumcases
\usepackage{enumitem}

\newcommand{\labfig}[1]{\label{fig:#1}}
\newcommand{\labtab}[1]{\label{tab:#1}}
\newcommand{\labeq}[1]{\label{eq:#1}}
\newcommand{\labsec}[1]{\label{sec:#1}}
\newcommand{\labchap}[1]{\label{chap:#1}}
\newcommand{\labitem}[1]{\label{item:#1}}
\newcommand{\figlab}[1]{\labfig{#1}} % alias
\newcommand{\tablab}[1]{\labtab{#1}} % alias
\newcommand{\eqlab}[1]{\labeq{#1}} % alias
\newcommand{\eqlabel}[1]{\labeq{#1}} % alias
\newcommand{\equlab}[1]{\labeq{#1}} % alias
\newcommand{\reffig}[1]{{図~\ref{#1}}~}
\newcommand{\reftab}[1]{{Table~\ref{#1}}~}
\newcommand{\refeq}[1]{{式~(\ref{#1})}~}
\newcommand{\refchap}[1]{第\ref{#1}章}
\newcommand{\refitem}[1]{\ref{#1}}
\newcommand{\refsec}[1]{第\ref{#1}節}
\newcommand{\figref}[1]{\reffig{#1}} % alias
\newcommand{\tabref}[1]{\reftab{#1}} % alias
\renewcommand{\eqref}[1]{\refeq{#1}} % alias
\newcommand{\chapref}[1]{\refchap{#1}} % alias
\newcommand{\secref}[1]{\refsec{#1}} % alias
\newcommand{\bm}[1]{\mbox{\boldmath{$#1$}}}
\makeatletter
\newcommand{\footnoteref}[1]{\protected@xdef\@thefnmark{\ref{#1}}\@footnotemark}
\@addtoreset{equation}{section}
\def\theequation{\thesection.\arabic{equation}}
\makeatother
\newcommand{\eqdef}{\ensuremath{\stackrel{\mathrm{def}}{=}}}
\newcommand{\argmax}{\mathop{\rm arg~max}\limits}
\newcommand{\argmin}{\mathop{\rm arg~min}\limits}


\begin{document}

\newcommand{\ptext}[1]
{\tt \begin{quote} \begin{tabbing} #1 \end{tabbing} \end{quote} \rm}

\newcommand{\desclist}[1]{
\begin{list}{ }{\setlength{\rightmargin}{0mm}\topsep=0mm\partopsep=0mm}
\item #1
\end{list}
\vspace{3mm}}

\newcommand{\functiondescription}[4]{
\index{#1}
{\bf #1} \em #2 \rm \hfill [#3] 
%\if#4 \vspace{3mm} \\ \else \desclist{#4} \fi
%\ifx#4 \vspace{3mm} \\ \else \desclist{#4} \fi
 \desclist{\hspace{0mm}#4}
}

\newcommand{\bfx}[1]{\index{#1}{\bf #1}}
\newcommand{\emx}[1]{\index{#1}{\em #1}}

\newcommand{\longdescription}[3]{
\index{#1}
\begin{emtabbing}
{\bf #1} 
\it #2
\rm
\end{emtabbing}
\desclist{#3}
}

\newcommand{\funcdesc}[3]{\functiondescription{#1}{#2}{function}{#3}}
\newcommand{\macrodesc}[3]{\functiondescription{#1}{#2}{macro}{#3}}
\newcommand{\specialdesc}[3]{\functiondescription{#1}{#2}{special}{#3}}
\newcommand{\methoddesc}[3]{\functiondescription{#1}{#2}{method}{#3}}
\newcommand{\vardesc}[2]{\functiondescription{#1}{}{変数}{#2}}

\newcommand{\fundesc}[2]{\functiondescription{#1}{#2}{function}{\hspace{0mm}}}
\newcommand{\macdesc}[2]{\functiondescription{#1}{#2}{macro}{\hspace{0mm}}}
\newcommand{\spedesc}[2]{\functiondescription{#1}{#2}{special}{\hspace{0mm}}}
\newcommand{\metdesc}[2]{\functiondescription{#1}{#2}{method}{\hspace{0mm}}}

\newcommand{\constdesc}[2]{\functiondescription{#1}{}{定数}{#2}}

\newcommand{\classdesc}[4]{	%class, super slots description
\vspace{2mm} 
\index{#1}
{\Large {\bf #1 }} \hfill [class]  %super
\begin{tabbing}
\hspace{30mm} :super \hspace{5mm} \= {\bf #2} \\
\hspace{30mm} :slots \> #3 
\end{tabbing}
\vspace{4mm}
\desclist{#4}}

\newenvironment{refdesc}{
 \vspace{5mm} \parindent=0mm \topsep=0mm \parskip=0mm \leftmargin=10mm}{
             \parindent=10mm \topsep=3mm \parskip=1mm \leftmargin=0mm }


\date{}
\title{{\LARGE \bf 軌道最適化による動作生成 \\ リファレンスマニュアル} \\
\vspace{10mm}
{\large \today} \\
}

\author{
室岡雅樹 \\
murooka@jsk.t.u-tokyo.ac.jp \\
}

\thispagestyle{empty}
\maketitle
\pagenumbering{roman}
\tableofcontents

\newpage
\pagenumbering{arabic}

%%%%%%%%%%%%%%%%%%%%%%
\section{軌道最適化による動作生成の基礎} \label{chap:fundamental}
\input{fundamental}

%%%%%%%%%%%%%%%%%%%%%%
\section{コンフィギュレーションとタスク関数} \label{chap:config-task}
%%%%%
\subsection{瞬時コンフィギュレーションと瞬時タスク関数} \label{sec:instant-config-task}
\input{instant-configuration-task}
%%%%%
\subsection{軌道コンフィギュレーションと軌道タスク関数} \label{sec:trajectory-config-task}
\input{trajectory-configuration-task}
%%%%%
\subsection{複合コンフィギュレーションと複合タスク関数} \label{sec:compound-config-task}
\input{compound-configuration-task}

%%%%%%%%%%%%%%%%%%%%%%
\section{勾配を用いた制約付き非線形最適化} \label{chap:sqp}
%%%%%
\subsection{逐次二次計画法} \label{sec:sqp}
\input{sqp-optimization}
%%%%%
\subsection{複数解候補を用いた逐次二次計画法} \label{sec:sqp-msc}
\subsubsection{複数解候補を用いた逐次二次計画法の理論}
\eqref{eq:ik-opt-3a}の最適化問題に逐次二次計画法などの制約付き非線形最適化手法を適用すると,
初期値から勾配方向に進行して至る局所最適解が得られると考えられる.
したがって解は初期値に強く依存する.

\eqref{eq:ik-opt-3a}の代わりに,以下の最適化問題を考える.
\begin{eqnarray}
  &&\min_{\bm{\hat{q}}} \ \sum_{i \in \mathcal{I}} \left\{ F(\bm{q}^{(i)}) + k_{\mathit{msc}} F_{\mathit{msc}}(\bm{\hat{q}}; i) \right\} \label{eq:solution-candidate-opt} \\
  &&{\rm s.t.} \ \  \bm{A} \bm{q}^{(i)} = \bm{\bar{b}} \ \ \ \ i \in \mathcal{I} \\
  &&\phantom{\rm s.t.} \ \  \bm{C} \bm{q}^{(i)} \geq \bm{\bar{d}} \ \ \ \ i \in \mathcal{I} \\
  &&{\rm where} \ \ \bm{\hat{q}} \eqdef \begin{pmatrix} \bm{q}^{(1)T} & \bm{q}^{(2)T} & \cdots & \bm{q}^{(N_{\mathit{msc}})T} \end{pmatrix}^T \\
  &&\phantom{\rm where} \ \ \mathcal{I} \eqdef \{ 1,2,\cdots,N_{\mathit{msc}} \} \\
  &&\phantom{\rm where} \ \ F_{\mathit{msc}}(\bm{\hat{q}}; i) \eqdef - \frac{1}{2} \sum_{\substack{j \in \mathcal{I} \\ j \not= i}} \log \| \bm{d}(\bm{q}^{(i)}, \bm{q}^{(j)}) \|^2 \\
  &&\phantom{\rm where} \ \ \bm{d}(\bm{q}^{(i)}, \bm{q}^{(j)}) \eqdef \bm{p}(\bm{q}^{(i)}) - \bm{p}(\bm{q}^{(j)})
\end{eqnarray}
$N_{\mathit{msc}}$は解候補の個数で,事前に与えるものとする.$\mathit{msc}$は複数解候補(multiple solution candidates)を表す.
これは,複数の解候補を同時に探索し,それぞれの解候補$\bm{q}^{(i)}$が本来の目的関数$F(\bm{q}^{(i)})$を小さくして,なおかつ,解候補どうしの距離が大きくなるように最適化することを表している
\footnote{$\bm{p}(\bm{q})$は,コンフィギュレーションを距離計算のために変換する関数である.
普通の場合は,コンフィギュレーションそのものが離れるように$\bm{p}(\bm{q})=\bm{q}$とすれば良い.}.
これにより,初期値に依存した唯一の局所解だけでなく,そこから離れた複数の局所解を得ることが可能となり,通常の最適化に比べてより良い解が得られることが期待される.
以降では,解候補どうしの距離のコストを表す項$F_{\mathit{msc}}(\bm{\hat{q}}; i)$を解候補分散項と呼ぶ
\footnote{解分散項の$\log$を無くすことは適切ではない.なぜなら,$d = \| \bm{d}(\bm{q}^{(i)}, \bm{q}^{(j)}) \|$として,解分散項の勾配は,
\begin{eqnarray}
  \frac{\partial}{\partial d}\left(- \frac{1}{2} \log d^2 \right) = - \frac{1}{d} \to - \infty \ \ (d \to +0) \hspace{10mm}
  \frac{\partial}{\partial d}\left(- \frac{1}{2} \log d^2 \right) = - \frac{1}{d} \to 0 \ \ (d \to \infty)
\end{eqnarray}
となり,最適化により,コンフィギュレーションが近いときほど離れるように更新し,遠くなるとその影響が小さくなる効果が期待される.それに対し,$\log$がない場合の勾配は,
\begin{eqnarray}
  \frac{\partial}{\partial d}\left(- \frac{1}{2} d^2 \right) = - d \to 0 \ \ (d \to +0) \hspace{10mm}
  \frac{\partial}{\partial d}\left(- \frac{1}{2} d^2 \right) = - d \to - \infty \ \ (d \to \infty)
\end{eqnarray}
となり,コンフィギュレーションが遠くなるほど離れるように更新し,近いときはその影響が小さくなる.これは,コンフィギュレーションが一致する勾配ゼロの点と,無限に離れ発散する最適値をもち,これらは最適化において望まない挙動をもたらす.
}.

解候補分散項のヤコビ行列,ヘッセ行列の各成分は次式で得られる\footnote{ヘッセ行列の導出は以下を参考にした.\url{https://math.stackexchange.com/questions/175263/gradient-and-hessian-of-general-2-norm}}.
\begin{subequations}
\begin{eqnarray}
  \nabla_i F_{\mathit{msc}}(\bm{\hat{q}}; i) &=& \frac{\partial F_{\mathit{msc}}(\bm{\hat{q}}; i)}{\partial \bm{q}^{(i)}} \\
  &=& - \frac{1}{2} \sum_{\substack{j \in \mathcal{I} \\ j \not= i}} \frac{\partial}{\partial \bm{q}^{(i)}} \log \| \bm{d}(\bm{q}^{(i)}, \bm{q}^{(j)}) \|^2 \\
  &=& - \sum_{\substack{j \in \mathcal{I} \\ j \not= i}} \frac{1}{\| \bm{d}(\bm{q}^{(i)}, \bm{q}^{(j)}) \|^2} \left( \frac{\partial \bm{d}(\bm{q}^{(i)}, \bm{q}^{(j)})}{\partial \bm{q}^{(i)}} \right)^T \bm{d}(\bm{q}^{(i)}, \bm{q}^{(j)})\\
  &=& - \sum_{\substack{j \in \mathcal{I} \\ j \not= i}} \frac{1}{\| \bm{d}(\bm{q}^{(i)}, \bm{q}^{(j)}) \|^2} {\bm{J}^{(i)}}^T \bm{d}(\bm{q}^{(i)}, \bm{q}^{(j)})\\
  &=& - \sum_{\substack{j \in \mathcal{I} \\ j \not= i}} \bm{g}(\bm{q}^{(j)}; i)
\end{eqnarray}
\end{subequations}
ただし,
\begin{eqnarray}
  \bm{J}^{(i)} &=& \left. \frac{\partial \bm{p}}{\partial \bm{q}} \right|_{\bm{q}=\bm{q}^{(i)}} \\
  \bm{g}(\bm{q}; i) &=& \frac{1}{\| \bm{d}(\bm{q}^{(i)}, \bm{q}) \|^2} {\bm{J}^{(i)}}^T \bm{d}(\bm{q}^{(i)}, \bm{q})
\end{eqnarray}
\begin{subequations}
\begin{eqnarray}
  \nabla_k F_{\mathit{msc}}(\bm{\hat{q}}; i) &=& \frac{\partial F_{\mathit{msc}}(\bm{\hat{q}}; i)}{\partial \bm{q}^{(k)}} \ \ \ \ k \in \mathcal{I} \land k \not= i \\
  &=& - \frac{1}{2} \sum_{\substack{j \in \mathcal{I} \\ j \not= i}} \frac{\partial}{\partial \bm{q}^{(k)}} \log \| \bm{d}(\bm{q}^{(i)}, \bm{q}^{(j)}) \|^2 \\
  &=& - \frac{1}{2} \frac{\partial}{\partial \bm{q}^{(k)}} \log \| \bm{d}(\bm{q}^{(i)}, \bm{q}^{(k)}) \|^2 \\
  &=& - \frac{1}{\| \bm{d}(\bm{q}^{(i)}, \bm{q}^{(k)}) \|^2} \left( \frac{\partial \bm{d}(\bm{q}^{(i)}, \bm{q}^{(k)})}{\partial \bm{q}^{(k)}} \right)^T \bm{d}(\bm{q}^{(i)}, \bm{q}^{(k)})\\
  &=& \frac{1}{\| \bm{d}(\bm{q}^{(i)}, \bm{q}^{(k)}) \|^2} {\bm{J}^{(k)}}^T \bm{d}(\bm{q}^{(i)}, \bm{q}^{(k)}) \\
  &=& - \frac{1}{\| \bm{d}(\bm{q}^{(k)}, \bm{q}^{(i)}) \|^2} {\bm{J}^{(k)}}^T \bm{d}(\bm{q}^{(k)}, \bm{q}^{(i)}) \\
  &=& - \bm{g}(\bm{q}^{(i)}; k)
\end{eqnarray}
\end{subequations}
\begin{subequations}
\begin{eqnarray}
  \nabla^2_{ii} F_{\mathit{msc}}(\bm{\hat{q}}; i) &=& \frac{\partial^2 F_{\mathit{msc}}(\bm{\hat{q}}; i)}{\partial \bm{q}^{(i) 2}} \\
  &=& - \sum_{\substack{j \in \mathcal{I} \\ j \not= i}} \frac{\partial}{\partial \bm{q}^{(i)}} \left( \left\{ \| \bm{d}(\bm{q}^{(i)}, \bm{q}^{(j)}) \|^2 \right\}^{-1} {\bm{J}^{(i)}}^T \bm{d}(\bm{q}^{(i)}, \bm{q}^{(j)}) \right) \\
  &\approx& - \sum_{\substack{j \in \mathcal{I} \\ j \not= i}} \left( - 2 \left\{ \| \bm{d}(\bm{q}^{(i)}, \bm{q}^{(j)}) \|^2 \right\}^{-2} {\bm{J}^{(i)}}^T \bm{d}(\bm{q}^{(i)}, \bm{q}^{(j)}) \bm{d}(\bm{q}^{(i)}, \bm{q}^{(j)})^T \bm{J}^{(i)} + \left\{ \| \bm{d}(\bm{q}^{(i)}, \bm{q}^{(j)}) \|^2 \right\}^{-1} {\bm{J}^{(i)}}^T \bm{J}^{(i)} \right) \\
  &=& - \sum_{\substack{j \in \mathcal{I} \\ j \not= i}} \left( - \frac{2}{\| \bm{d}(\bm{q}^{(i)}, \bm{q}^{(j)}) \|^4} {\bm{J}^{(i)}}^T \bm{d}(\bm{q}^{(i)}, \bm{q}^{(j)}) \bm{d}(\bm{q}^{(i)}, \bm{q}^{(j)})^T \bm{J}^{(i)} + \frac{1}{ \| \bm{d}(\bm{q}^{(i)}, \bm{q}^{(j)}) \|^2 } {\bm{J}^{(i)}}^T \bm{J}^{(i)} \right) \\
  &=& - \sum_{\substack{j \in \mathcal{I} \\ j \not= i}} \bm{H}(\bm{q}^{(i)}, \bm{q}^{(j)}; i, i)
\end{eqnarray}
\end{subequations}
ただし,
\begin{eqnarray}
  \bm{H}(\bm{q}_1, \bm{q}_2; i, j) \eqdef
  - \frac{2}{\| \bm{d}(\bm{q}_1, \bm{q}_2) \|^4} {\bm{J}^{(i)}}^T \bm{d}(\bm{q}_1, \bm{q}_2) \bm{d}(\bm{q}_1, \bm{q}_2)^T \bm{J}^{(j)} + \frac{1}{ \| \bm{d}(\bm{q}_1, \bm{q}_2) \|^2 } {\bm{J}^{(i)}}^T \bm{J}^{(j)}
\end{eqnarray}
また,
距離計算のための変換関数の二階微分$\frac{\partial \bm{J}^{(i)}}{\partial \bm{q}^{(i)}}$をゼロ行列として近似した.
\begin{subequations}
\begin{eqnarray}
  \nabla^2_{ik} F_{\mathit{msc}}(\bm{\hat{q}}; i) &=& \frac{\partial^2 F_{\mathit{msc}}(\bm{\hat{q}}; i)}{\partial \bm{q}^{(i)} \partial \bm{q}^{(k)}} \ \ \ \ k \in \mathcal{I} \land k \not= i \\
  &=& - \sum_{\substack{j \in \mathcal{I} \\ j \not= i}} \frac{\partial}{\partial \bm{q}^{(k)}} \left( \left\{ \| \bm{d}(\bm{q}^{(i)}, \bm{q}^{(j)}) \|^2 \right\}^{-1} {\bm{J}^{(i)}}^T \bm{d}(\bm{q}^{(i)}, \bm{q}^{(j)}) \right) \\
  &=& - \frac{\partial}{\partial \bm{q}^{(k)}} \left( \left\{ \| \bm{d}(\bm{q}^{(i)}, \bm{q}^{(k)}) \|^2 \right\}^{-1} {\bm{J}^{(i)}}^T \bm{d}(\bm{q}^{(i)}, \bm{q}^{(k)}) \right) \\
  &\approx& - \left( 2 \left\{ \| \bm{d}(\bm{q}^{(i)}, \bm{q}^{(k)}) \|^2 \right\}^{-2} {\bm{J}^{(k)}}^T \bm{d}(\bm{q}^{(i)}, \bm{q}^{(k)}) \bm{d}(\bm{q}^{(i)}, \bm{q}^{(k)})^T \bm{J}^{(i)} - \left\{ \| \bm{d}(\bm{q}^{(i)}, \bm{q}^{(k)}) \|^2 \right\}^{-1} {\bm{J}^{(i)}}^T \bm{J}^{(k)} \right) \\
  &=& - \frac{2}{\| \bm{d}(\bm{q}^{(i)}, \bm{q}^{(k)}) \|^4} {\bm{J}^{(k)}}^T \bm{d}(\bm{q}^{(i)}, \bm{q}^{(k)}) \bm{d}(\bm{q}^{(i)}, \bm{q}^{(k)})^T {\bm{J}^{(i)}} + \frac{1}{ \| \bm{d}(\bm{q}^{(i)}, \bm{q}^{(k)}) \|^2 } {\bm{J}^{(i)}}^T \bm{J}^{(k)} \\
  &=& \bm{H}(\bm{q}^{(i)}, \bm{q}^{(k)}; k, i) \\
  &=& \bm{H}(\bm{q}^{(k)}, \bm{q}^{(i)}; k, i)
\end{eqnarray}
\end{subequations}
\begin{subequations}
\begin{eqnarray}
  \nabla^2_{kk} F_{\mathit{msc}}(\bm{\hat{q}}; i) &=& \frac{\partial^2 F_{\mathit{msc}}(\bm{\hat{q}}; i)}{\partial \bm{q}^{(k)2}} \ \ \ \ k \in \mathcal{I} \land k \not= i\\
  &=& - \frac{\partial}{\partial \bm{q}^{(k)}} \left( \left\{ \| \bm{d}(\bm{q}^{(k)}, \bm{q}^{(i)}) \|^2 \right\}^{-1} {\bm{J}^{(k)}}^T \bm{d}(\bm{q}^{(k)}, \bm{q}^{(i)}) \right) \\
  &\approx& - \left( - \frac{2}{\| \bm{d}(\bm{q}^{(k)}, \bm{q}^{(i)}) \|^4} {\bm{J}^{(k)}}^T \bm{d}(\bm{q}^{(k)}, \bm{q}^{(i)}) \bm{d}(\bm{q}^{(k)}, \bm{q}^{(i)})^T \bm{J}^{(k)} + \frac{1}{ \| \bm{d}(\bm{q}^{(k)}, \bm{q}^{(i)}) \|^2 } {\bm{J}^{(k)}}^T \bm{J}^{(k)} \right) \\
  &=& - \bm{H}(\bm{q}^{(k)}, \bm{q}^{(i)}; k, k)
\end{eqnarray}
\end{subequations}
\begin{subequations}
\begin{eqnarray}
  \nabla^2_{kl} F_{\mathit{msc}}(\bm{\hat{q}}; i) &=& \frac{\partial^2 F_{\mathit{msc}}(\bm{\hat{q}}; i)}{\partial \bm{q}^{(k)} \partial \bm{q}^{(l)}} \ \ \ \ k \in \mathcal{I} \land l \in \mathcal{I} \land k \not= i \land l \not= i \land k \not= l\\
  &=& - \frac{\partial}{\partial \bm{q}^{(l)}} \left( \left\{ \| \bm{d}(\bm{q}^{(k)}, \bm{q}^{(i)}) \|^2 \right\}^{-1} {\bm{J}^{(k)}}^T \bm{d}(\bm{q}^{(k)}, \bm{q}^{(i)}) \right) \\
  &=& \bm{O}
\end{eqnarray}
\end{subequations}
したがって,
解候補分散項のヤコビ行列,ヘッセ行列は次式で表される.
\begin{subequations}
\begin{eqnarray}
  \nabla F_{\mathit{msc}}(\bm{\hat{q}}; i) &=& \frac{\partial F_{\mathit{msc}}(\bm{\hat{q}}; i)}{\partial \bm{\hat{q}}} \\
  &=&
  \begin{pmatrix}
  - \bm{g}(\bm{q}^{(i)}; 1) \\
  \vdots \\
  - \bm{g}(\bm{q}^{(i)}; i-1) \\
  - \sum_{\substack{j \in \mathcal{I} \\ j \not= i}} \bm{g}(\bm{q}^{(j)}; i) \\
  - \bm{g}(\bm{q}^{(i)}; i+1) \\
  \vdots \\
  - \bm{g}(\bm{q}^{(i)}; N_{\mathit{msc}})
  \end{pmatrix} \\
  \bm{v}_{\mathit{msc}} &\eqdef&
  \sum_{i \in \mathcal{I}} \nabla F_{\mathit{msc}}(\bm{\hat{q}}; i) \\
  &=&
  - 2
  \begin{pmatrix}
  \sum_{\substack{j \in \mathcal{I} \\ j \not= 1}} \bm{g}(\bm{q}^{(j)}; 1) \\
  \vdots \\
  \sum_{\substack{j \in \mathcal{I} \\ j \not= N_{\mathit{msc}}}} \bm{g}(\bm{q}^{(j)}; N_{\mathit{msc}}) \\
  \end{pmatrix} \label{eq:sqp-msc-dispersion-matrix}
\end{eqnarray}
\end{subequations}
\begin{subequations}
\begin{eqnarray}
  \nabla^2 F_{\mathit{msc}}(\bm{\hat{q}}; i) &=& \frac{\partial^2 F_{\mathit{msc}}(\bm{\hat{q}}; i)}{\partial \bm{\hat{q}}^2} \\
  &=&
  \bordermatrix{
    & 1 & \cdots & i-1 & i & i+1 & \cdots & N_{\mathit{msc}} \cr
    1 & - \bm{H}_{i,1} & & & \bm{H}_{i,1} & & &\cr
    \vdots & & \ddots & & \vdots & & & \cr
    i-1 & & & - \bm{H}_{i,i-1} & \bm{H}_{i,i-1} & & & \cr
    i & \bm{H}_{i,1} & \cdots & \bm{H}_{i,i-1} & - \sum_{\substack{j \in \mathcal{I} \\ j \not= i}} \bm{H}_{i,j} & \bm{H}_{i,i+1} & \cdots & \bm{H}_{i,N_{\mathit{msc}}} \cr
    i+1 & & & & \bm{H}_{i,i+1} & - \bm{H}_{i,i+1} & & \cr
    \vdots & & & & \vdots & & \ddots & \cr
    N_{\mathit{msc}} & & & & \bm{H}_{i,N_{\mathit{msc}}} & & & - \bm{H}_{i, N_{\mathit{msc}}} \cr
  } \\
  \bm{W}_{\mathit{msc}} &\eqdef&
  \sum_{i \in \mathcal{I}} \nabla^2 F_{\mathit{msc}}(\bm{\hat{q}}; i) \\
  &=&
  2
  \begin{pmatrix}
    - \sum_{\substack{j \in \mathcal{I} \\ j \not= i}} \bm{H}_{1,j} & \bm{H}_{1,2} & \cdots & \bm{H}_{1,N_{\mathit{msc}}} \\
    \bm{H}_{2,1} & - \sum_{\substack{j \in \mathcal{I} \\ j \not= i}} \bm{H}_{2,j} & & \bm{H}_{2,N_{\mathit{msc}}} \\
    \vdots & & \ddots & \vdots \\
    \bm{H}_{N_{\mathit{msc}},1} & \bm{H}_{N_{\mathit{msc}},2} & \cdots & - \sum_{\substack{j \in \mathcal{I} \\ j \not= i}} \bm{H}_{N_{\mathit{msc}},j}
  \end{pmatrix} \label{eq:sqp-msc-dispersion-vector}
\end{eqnarray}
\end{subequations}
ただし,$\bm{H}(\bm{q}^{(i)}, \bm{q}^{(j)})$を$\bm{H}_{i,j}$と略して記す.
また,$\bm{d}(\bm{q}^{(i)}, \bm{q}^{(j)}) = - \bm{d}(\bm{q}^{(j)}, \bm{q}^{(i)}), \ \bm{H}_{i,j} = \bm{H}_{j,i}$を利用した.

解候補分散項$\sum_{i \in \mathcal{I}} F_{\mathit{msc}}(\bm{\hat{q}}; i)$による二次計画問題の目的関数(\eqref{eq:sqp-1a})は次式で表される.
\begin{eqnarray}
  &&\sum_{i \in \mathcal{I}} \left\{ F_{\mathit{msc}}(\bm{\hat{q}}_k; i) + \nabla F_{\mathit{msc}}(\bm{\hat{q}}_k; i)^T \Delta \bm{\hat{q}}_k + \frac{1}{2} \Delta \bm{\hat{q}}_k^T \nabla^2 F_{\mathit{msc}}(\bm{\hat{q}}_k; i) \Delta \bm{\hat{q}}_k \right\} \\
  &=&
  \sum_{i \in \mathcal{I}} F_{\mathit{msc}}(\bm{\hat{q}}_k; i)
  + \left\{ \sum_{i \in \mathcal{I}} \nabla F_{\mathit{msc}}(\bm{\hat{q}}_k; i) \right\}^T \Delta \bm{\hat{q}}_k
  + \frac{1}{2} \Delta \bm{\hat{q}}_k^T \left\{ \sum_{i \in \mathcal{I}} \nabla^2 F_{\mathit{msc}}(\bm{\hat{q}}_k; i) \right\} \Delta \bm{\hat{q}}_k \\
  &=&
  \sum_{i \in \mathcal{I}} F_{\mathit{msc}}(\bm{\hat{q}}_k; i) + \bm{v}_{\mathit{msc}}^T \Delta \bm{\hat{q}}_k + \frac{1}{2} \Delta \bm{\hat{q}}_k^T \bm{W}_{\mathit{msc}} \Delta \bm{\hat{q}}_k
\end{eqnarray}

$\bm{W}_{\mathit{msc}}$が必ずしも半正定値行列ではないことに注意する必要がある.
以下のようにして$\bm{W}_{\mathit{msc}}$に近い正定値行列を計算し用いることで対処する
\footnote{$\bm{W}_{\mathit{msc}}$が対称行列であることから,以下を参考にした.\url{https://math.stackexchange.com/questions/648809/how-to-find-closest-positive-definite-matrix-of-non-symmetric-matrix\#comment1689831_649522}}.
$\bm{W}_{\mathit{msc}}$が次式のように固有値分解されるとする.
\begin{eqnarray}
  \bm{W}_{\mathit{msc}} = \bm{V}_{\mathit{msc}} \bm{D}_{\mathit{msc}} \bm{V}_{\mathit{msc}}^{-1}
\end{eqnarray}
ただし,$\bm{D}_{\mathit{msc}}$は固有値を対角成分にもつ対角行列,$\bm{V}_{\mathit{msc}}$は固有ベクトルを並べた行列である.
このとき$\bm{W}_{\mathit{msc}}$に近い正定値行列$\bm{\tilde{W}}_{\mathit{msc}}$は次式で得られる.
\begin{eqnarray}
  \bm{\tilde{W}}_{\mathit{msc}} = \bm{V}_{\mathit{msc}} \bm{D}_{\mathit{msc}}^+ \bm{V}_{\mathit{msc}}^{-1}
\end{eqnarray}
ただし,$\bm{D}_{\mathit{msc}}^+$は$\bm{D}_{\mathit{msc}}$の対角成分のうち,負のものを$0$で置き換えた対角行列である.

\eqref{eq:solution-candidate-opt}において,
解候補を分散させながら,最終的に本来の目的関数を最小にする解を得るために,
SQPのイテレーションごとに,解候補分散項のスケール$k_{\mathit{msc}}$を次式のように更新することが有効である.
\begin{eqnarray}
  k_{\mathit{msc}} \gets \min ( \gamma_{\mathit{msc}} k_{\mathit{msc}}, k_{\mathit{msc\mathchar`-min}})
\end{eqnarray}
$\gamma_{\mathit{msc}}$は$0< \gamma_{\mathit{msc}} < 1$なるスケール減少率,
$k_{\mathit{msc\mathchar`-min}}$はスケール最小値を表す.

\subsubsection{複数解候補を用いた逐次二次計画法の実装}
\input{sqp-msc-optimization}

%%%%%%%%%%%%%%%%%%%%%%
\section{動作生成の拡張} \label{chap:extended}
%%%%%
\subsection{マニピュレーションの動作生成} \label{sec:manip}
\input{robot-object-environment}
\input{instant-manipulation-configuration-task}
%%%%%
\subsection{Bスプラインを用いた関節軌道生成} \label{sec:bspline}
\subsubsection{Bスプラインを用いた関節軌道生成の理論}
\input{theory-bspline}
\subsubsection{Bスプラインを用いた関節軌道生成の実装}
\input{bspline-configuration-task}
%%%%%
\subsection{Bスプラインを用いた動的動作の生成} \label{sec:dynamic}
\input{bspline-dynamic-configuration-task}
%%%%%
\subsection{離散的な幾何目標に対する逆運動学計算} \label{sec:discrete-ik}
\subsubsection{離散的な幾何目標に対する逆運動学計算の理論}
%%
\subsubsection*{min/max関数の微分可能関数近似}

minimum/maximum関数
\begin{eqnarray}
  F_{min}(\bm{x}; f_1, \cdots, f_K) &\eqdef& \min (f_1(\bm{x}), \cdots, f_K(\bm{x})) \label{eq:original-min} \\
  F_{max}(\bm{x}; f_1, \cdots, f_K) &\eqdef& \max (f_1(\bm{x}), \cdots, f_K(\bm{x})) \label{eq:original-max}
\end{eqnarray}
を連続かつ微分可能な関数で近似したsmooth minimum/maximum関数として,次式を用いることができる\footnote{\url{https://en.wikipedia.org/wiki/Smooth_maximum}}.
\begin{eqnarray}
  \mathcal{S}_{\alpha}(\bm{x}; f_1, \cdots, f_K) \eqdef \frac{\sum_{k=1}^{K} f_k(\bm{x}) e^{\alpha f_k(\bm{x})}}{\sum_{k=1}^{K} e^{\alpha f_k(\bm{x})}} \label{eq:smooth-max}
\end{eqnarray}
この関数は以下の性質をもつ.
\begin{eqnarray}
  \alpha \to - \inf &のとき& \mathcal{S}_{\alpha} \to F_{min} \\
  \alpha \to \inf &のとき& \mathcal{S}_{\alpha} \to F_{max}
\end{eqnarray}

%%
\subsubsection*{離散的な目標に対するタスク関数の微分可能関数近似}

タスク関数として
$\bm{e}_1(\bm{q}), \cdots, \bm{e}_K(\bm{q}) \in \mathbb{R}^{N_e}$が与えられているときに,
これらのタスク関数のいずれかをゼロにするコンフィギュレーション$\bm{q} \in \mathbb{R}^{N_q}$を求める問題を考える.
複数個の目標位置のいずれかにリーチングする逆運動学問題などがこの問題に含まれる.

この問題は次式で表される.
\begin{eqnarray}
  \bm{e}_k(\bm{q}) = \bm{0} \ \ (kは1,\cdots,Kのいずれか)
\end{eqnarray}
これは次式と同値である.
\begin{eqnarray}
  &&\bm{e}_{min}(\bm{q}) = \bm{0} \\
  &&{\rm where \ \ } \bm{e}_{min}(\bm{q}) \eqdef \argmin_{\bm{e}_k \in \mathcal{E}} \| \bm{e}_k(\bm{q}) \|^2 \in \mathbb{R}^{N_e} \\
  &&\phantom{\rm where \ \ }\mathcal{E} \eqdef \{ \bm{e}_1,\cdots,\bm{e}_K \}
\end{eqnarray}
タスク関数$\bm{e}_{min}(\bm{q})$のヤコビ行列$\frac{\partial \bm{e}_{min}(\bm{q})}{\partial \bm{q}}$が導出できれば,
\chapref{chap:fundamental}の定式化により最適化計算を行うことで
コンフィギュレーション$\bm{q}$を求めることができる.
しかし,$\bm{e}_{min}(\bm{q})$は一般に,最小の$\bm{e}_k$が切り替わる点において微分不可能であり,
ヤコビ行列を求めることができない.

\eqref{eq:smooth-max}では,
$f_k(\bm{x}) \in \mathbb{R} \ (k=1,\cdots,K)$
の
$\dfrac{e^{\alpha f_k(\bm{x})}}{\sum_{k=1}^{K} e^{\alpha f_k(\bm{x})}}$
による重み付けした和をとることで,
min/maxの微分可能関数近似を得ている.
この近似をスカラ値関数からベクトル値関数へと拡張して,
$\bm{e}_{min}(\bm{q})$を次式の微分可能関数で近似する.
\begin{eqnarray}
  &&\bm{\hat{e}}_{min}(\bm{q}) \eqdef
  \dfrac{1}{ \sum_{\bm{e}_k \in \mathcal{E}} \exp(-\alpha \| \bm{e}_k(\bm{q}) \|^2) }
  \sum_{\bm{e}_k \in \mathcal{E}} \exp(-\alpha \| \bm{e}_k(\bm{q}) \|^2) \bm{e}_k(\bm{q}) \in \mathbb{R}^{N_e} \label{eq:smooth-task-func}
\end{eqnarray}
$\alpha$は正の定数で大きいほど近似精度が増す.
タスク関数$\bm{\hat{e}}_{min}(\bm{q})$のヤコビ行列$\frac{\partial \bm{\hat{e}}_{min}(\bm{q})}{\partial \bm{q}}$は,
解析的に導出可能である.

%%
\subsubsection*{contact-invariant-optimizationにおける微分可能関数近似 (参考)}
contact-invariant-optimizationの論文
\footnote{
  Discovery of complex behaviors through contact-invariant optimization,
  I. Mordatch, et. al.,
  ACM Transactions on Graphics 31.4, 43, 2012.
}
の4.1節では,minimum関数を含むタスク関数が以下のように近似されている.
\begin{eqnarray}
  &&\bm{\hat{e}}_{min}(\bm{q}) \eqdef
  \dfrac{1}{ \sum_{\bm{e}_k \in \mathcal{E}} \eta(\bm{e}_k(\bm{q})) }
  \sum_{\bm{e}_k \in \mathcal{E}} \eta(\bm{e}_k(\bm{q})) \bm{e}_k(\bm{q}) \in \mathbb{R}^{N_e} \\
  &&{\rm where \ \ } \eta(\bm{e}_k(\bm{q})) = \frac{1}{1 + \beta \| \bm{e}_k(\bm{q}) \|^2} \in \mathbb{R}
\end{eqnarray}
$\beta$は正の定数で,論文では$10^4$としている.
これは,\eqref{eq:smooth-task-func}における
$\exp(-\alpha \| \bm{e}_k(\bm{q}) \|^2)$を$\eta(\bm{e}_k(\bm{q}))$で置き換えたものである.

%%
\subsubsection*{LogSumExpによる微分可能関数近似 (参考)}

\eqref{eq:original-min},\eqref{eq:original-max}のminimum/maximum関数
を連続かつ微分可能な関数で近似したsmooth minimum/maximum関数として,
LogSumExp関数を用いることができる\footnote{\url{https://en.wikipedia.org/wiki/Smooth_maximum}}.
\begin{eqnarray}
  LSE_{\varepsilon}(\bm{x}; f_1, \cdots, f_K) \eqdef \frac{\log \left( \sum_{k=1}^{K} \exp( \varepsilon f_k(\bm{x})) \right) }{\varepsilon} \label{eq:lse-smooth-max}
\end{eqnarray}
$\varepsilon$が負のときminimum関数,正のときmaximum関数の近似となり,絶対値が大きいほど近似精度が増す.

この関数は,重み付け和の形式ではないため,
\eqref{eq:smooth-task-func}のようにスカラ値関数からベクトル値関数へ拡張することができない.

タスク関数のノルム二乗として表される最適化の目的関数
\begin{eqnarray}
  F(\bm{q}) \eqdef \min_{\bm{e}_k \in \mathcal{E}} \| \bm{e}_k(\bm{q}) \|^2 \in \mathbb{R}
\end{eqnarray}
は,次の$\hat{F}(\bm{q})$として近似できる.
\begin{eqnarray}
  && \hat{F}(\bm{q}) \approx
  \frac{\log\left(\sum_{\bm{e}_k \in \mathcal{E}} \exp(- \varepsilon \| \bm{e}_k(\bm{q}) \|^2)\right)}{- \varepsilon}
\end{eqnarray}
\eqref{eq:lse-smooth-max}の$\varepsilon$を改めて$- \varepsilon$と置き直した.$\varepsilon$が大きいほど近似精度が増す.

近似目的関数$\hat{F}(\bm{q})$の勾配は次式で表される.
\begin{eqnarray}
  \frac{\partial \hat{F}(\bm{q})}{\partial \bm{q}} =
  \frac{\sum_{\bm{e}_k \in \mathcal{E}} 2 \varepsilon \exp(- \varepsilon \| \bm{e}_k(\bm{q}) \|^2) \left(\frac{\partial \bm{e}_k(\bm{q})}{\partial \bm{q}}\right)^T \bm{e}_k(\bm{q})}
       {\varepsilon \sum_{\bm{e}_k \in \mathcal{E}} \exp(- \varepsilon \| \bm{e}_k(\bm{q}) \|^2)}
\end{eqnarray}

近似目的関数$\hat{F}(\bm{q})$のヘッセ行列も解析的に導出可能である.
(タスク関数を考える場合,そのヤコビ行列が求まれば,\chapref{chap:fundamental}のように目的関数のヘッセ行列は導出可能である.しかし,今回のように目的関数を直接扱う場合は,そのヘッセ行列を陽に導出する必要がある.)

\subsubsection{離散的な幾何目標に対する逆運動学計算の実装}
\input{discrete-kinematics-configuration-task}

%%%%%%%%%%%%%%%%%%%%%%
\section{補足} \label{chap:appendix}
%%%%%
\subsection{既存のロボット基礎クラスの拡張} \label{sec:base-extention}
\input{extended-joint-link}
%%%%%
\subsection{環境と接触するロボットの関節・リンク構造} \label{sec:robot-environment}
\input{contact-kinematics}
\input{robot-environment}
%%%%%
\subsection{irteusのinverse-kinematics互換関数} \label{sec:ik-wrapper}
\input{inverse-kinematics-wrapper}
%%%%%
\subsection{関節トルク勾配の計算} \label{sec:torque-jacobian}
\input{torque-gradient}

\end{document}

